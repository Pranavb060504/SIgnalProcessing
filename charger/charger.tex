\documentclass[journal,12pt,twocolumn]{IEEEtran}
%
\usepackage{setspace}
\usepackage{gensymb}
\usepackage{xcolor}
\usepackage{caption}
%\usepackage{subcaption}
%\doublespacing
\singlespacing

%\usepackage{graphicx}
%\usepackage{amssymb}
%\usepackage{relsize}
\usepackage[cmex10]{amsmath}
\usepackage{mathtools}
%\usepackage{amsthm}
%\interdisplaylinepenalty=2500
%\savesymbol{iint}
%\usepackage{txfonts}
%\restoresymbol{TXF}{iint}
%\usepackage{wasysym}
\usepackage{hyperref}
\usepackage{amsthm}
\usepackage{mathrsfs}
\usepackage{txfonts}
\usepackage{stfloats}
\usepackage{cite}
\usepackage{cases}
\usepackage{subfig}
%\usepackage{xtab}
\usepackage{longtable}
\usepackage{multirow}
\usepackage{circuitikz}
%\usepackage{algorithm}
%\usepackage{algpseudocode}
%\usepackage{enumerate}
\usepackage{enumitem}
\usepackage{mathtools}
%\usepackage{iithtlc}
%\usepackage[framemethod=tikz]{mdframed}
\usepackage{listings}
\let\vec\mathbf


%\usepackage{stmaryrd}


%\usepackage{wasysym}
%\newcounter{MYtempeqncnt}
\DeclareMathOperator*{\Res}{Res}
%\renewcommand{\baselinestretch}{2}
\renewcommand\thesection{\arabic{section}}
\renewcommand\thesubsection{\thesection.\arabic{subsection}}
\renewcommand\thesubsubsection{\thesubsection.\arabic{subsubsection}}

\renewcommand\thesectiondis{\arabic{section}}
\renewcommand\thesubsectiondis{\thesectiondis.\arabic{subsection}}
\renewcommand\thesubsubsectiondis{\thesubsectiondis.\arabic{subsubsection}}

%\renewcommand{\labelenumi}{\textbf{\theenumi}}
%\renewcommand{\theenumi}{P.\arabic{enumi}}

% correct bad hyphenation here
\hyphenation{op-tical net-works semi-conduc-tor}

\lstset{
language=Python,
frame=single, 
breaklines=true,
columns=fullflexible
}



\begin{document}
%

\theoremstyle{definition}
\newtheorem{theorem}{Theorem}[section]
\newtheorem{problem}{Problem}
\newtheorem{proposition}{Proposition}[section]
\newtheorem{lemma}{Lemma}[section]
\newtheorem{corollary}[theorem]{Corollary}
\newtheorem{example}{Example}[section]
\newtheorem{definition}{Definition}[section]
%\newtheorem{algorithm}{Algorithm}[section]
%\newtheorem{cor}{Corollary}
\newcommand{\BEQA}{\begin{eqnarray}}
\newcommand{\EEQA}{\end{eqnarray}}
\newcommand{\define}{\stackrel{\triangle}{=}}
\newcommand{\myvec}[1]{\ensuremath{\begin{pmatrix}#1\end{pmatrix}}}
\newcommand{\mydet}[1]{\ensuremath{\begin{vmatrix}#1\end{vmatrix}}}
\bibliographystyle{IEEEtran}
%\bibliographystyle{ieeetr}
\providecommand{\nCr}[2]{\,^{#1}C_{#2}} % nCr
\providecommand{\nPr}[2]{\,^{#1}P_{#2}} % nPr
\providecommand{\mbf}{\mathbf}
\providecommand{\pr}[1]{\ensuremath{\Pr\left(#1\right)}}
\providecommand{\qfunc}[1]{\ensuremath{Q\left(#1\right)}}
\providecommand{\sbrak}[1]{\ensuremath{{}\left[#1\right]}}
\providecommand{\lsbrak}[1]{\ensuremath{{}\left[#1\right.}}
\providecommand{\rsbrak}[1]{\ensuremath{{}\left.#1\right]}}
\providecommand{\brak}[1]{\ensuremath{\left(#1\right)}}
\providecommand{\lbrak}[1]{\ensuremath{\left(#1\right.}}
\providecommand{\rbrak}[1]{\ensuremath{\left.#1\right)}}
\providecommand{\cbrak}[1]{\ensuremath{\left\{#1\right\}}}
\providecommand{\lcbrak}[1]{\ensuremath{\left\{#1\right.}}
\providecommand{\rcbrak}[1]{\ensuremath{\left.#1\right\}}}
\providecommand{\rect}[1]{\text{rect}\ensuremath{\left(#1\right)}}
\providecommand{\sinc}[1]{\text{sinc}\ensuremath{\left(#1\right)}}
\theoremstyle{remark}
\newtheorem{rem}{Remark}
\newcommand{\sgn}{\mathop{\mathrm{sgn}}}
\providecommand{\abs}[1]{\left\vert#1\right\vert}
\providecommand{\res}[1]{\Res\displaylimits_{#1}} 
\providecommand{\norm}[1]{\lVert#1\rVert}
\providecommand{\mtx}[1]{\mathbf{#1}}
\providecommand{\mean}[1]{E\left[ #1 \right]}
\providecommand{\fourier}{\overset{\mathcal{F}}{ \rightleftharpoons}}
\providecommand{\ztrans}{\overset{\mathcal{Z}}{ \rightleftharpoons}}
%\providecommand{\hilbert}{\overset{\mathcal{H}}{ \rightleftharpoons}}
%\providecommand{\system}{\overset{\mathcal{H}}{ \longleftrightarrow}}
\providecommand{\system}[1]{\overset{\mathcal{#1}}{ \longleftrightarrow}}
	%\newcommand{\solution}[2]{\textbf{Solution:}{#1}}
\newcommand{\solution}{\noindent \textbf{Solution: }}
\providecommand{\dec}[2]{\ensuremath{\overset{#1}{\underset{#2}{\gtrless}}}}
\numberwithin{equation}{section}
%\numberwithin{equation}{subsection}
%\numberwithin{problem}{subsection}
%\numberwithin{definition}{subsection}
\makeatletter
\@addtoreset{figure}{problem}
\makeatother
\let\StandardTheFigure\thefigure
%\renewcommand{\thefigure}{\theproblem.\arabic{figure}}
\renewcommand{\thefigure}{\theproblem}
%\numberwithin{figure}{subsection}
\def\putbox#1#2#3{\makebox[0in][l]{\makebox[#1][l]{}\raisebox{\baselineskip}[0in][0in]{\raisebox{#2}[0in][0in]{#3}}}}
     \def\rightbox#1{\makebox[0in][r]{#1}}
     \def\centbox#1{\makebox[0in]{#1}}
     \def\topbox#1{\raisebox{-\baselineskip}[0in][0in]{#1}}
     \def\midbox#1{\raisebox{-0.5\baselineskip}[0in][0in]{#1}}
\vspace{3cm}
\title{ 
%\logo{
%}
Fourier Series
%	\logo{Octave for Math Computing }
}
%\title{
%	\logo{Matrix Analysis through Octave}{\begin{center}\includegraphics[scale=.24]{tlc}\end{center}}{}{HAMDSP}
%}
% paper title
% can use linebreaks \\ within to get better formatting as desired
%\title{Matrix Analysis through Octave}
%
%
% author names and IEEE memberships
% note positions of commas and nonbreaking spaces ( ~ ) LaTeX will not break
% a structure at a ~ so this keeps an author's name from being broken across
% two lines.
% use \thanks{} to gain access to the first footnote area
% a separate \thanks must be used for each paragraph as LaTeX2e's \thanks
% was not built to handle multiple paragraphs
%
\author{ Pranav B$^{*}$ %<-this  stops a space
% <-this % stops a space
%\thanks{J. Doe and J. Doe are with Anonymous University.}% <-this % stops a space
%\thanks{Manuscript received April 19, 2005; revised January 11, 2007.}}
}
% note the % following the last \IEEEmembership and also \thanks - 
% these prevent an unwanted space from occurring between the last author name
% and the end of the author line. i.e., if you had this:
% 
% \author{....lastname \thanks{...} \thanks{...} }
%                     ^------------^------------^----Do not want these spaces!
%
% a space would be appended to the last name and could cause every name on that
% line to be shifted left slightly. This is one of those "LaTeX things". For
% instance, "\textbf{A} \textbf{B}" will typeset as "A B" not "AB". To get
% "AB" then you have to do: "\textbf{A}\textbf{B}"
% \thanks is no different in this regard, so shield the last } of each \thanks
% that ends a line with a % and do not let a space in before the next \thanks.
% Spaces after \IEEEmembership other than the last one are OK (and needed) as
% you are supposed to have spaces between the names. For what it is worth,
% this is a minor point as most people would not even notice if the said evil
% space somehow managed to creep in.
% The paper headers
%\markboth{Journal of \LaTeX\ Class Files,~Vol.~6, No.~1, January~2007}%
%{Shell \MakeLowercase{\textit{et al.}}: Bare Demo of IEEEtran.cls for Journals}
% The only time the second header will appear is for the odd numbered pages
% after the title page when using the twoside option.
% 
% *** Note that you probably will NOT want to include the author's ***
% *** name in the headers of peer review papers.                   ***
% You can use \ifCLASSOPTIONpeerreview for conditional compilation here if
% you desire.
% If you want to put a publisher's ID mark on the page you can do it like
% this:
%\IEEEpubid{0000--0000/00\$00.00~\copyright~2007 IEEE}
% Remember, if you use this you must call \IEEEpubidadjcol in the second
% column for its text to clear the IEEEpubid mark.
% make the title area
\maketitle
%\newpage
\tableofcontents
%\renewcommand{\thefigure}{\thesection.\theenumi}
%\renewcommand{\thetable}{\thesection.\theenumi}
\renewcommand{\thefigure}{\theenumi}
\renewcommand{\thetable}{\theenumi}
%\renewcommand{\theequation}{\thesection}
\bigskip
\begin{abstract}
This manual provides a simple introduction to Fourier Series
\end{abstract}
\section{Periodic Function}
Let 
\begin{align}
	x(t) &= A_0\abs{\sin\brak{2\pi f_0 t}}
	\label{eq:10-orig-diff-def}
\end{align}
\begin{enumerate}[label=\thesection.\arabic*
,ref=\thesection.\theenumi]
\item Plot $x(t)$.\\
    \solution
    		Run the following code 
\begin{lstlisting}
wget https://github.com/Pranavb060504/SIgnalProcessing/blob/main/charger/codes/f.py
\end{lstlisting}
Use the following command in the terminal to run the code
\begin{lstlisting}
python3 f.py
\end{lstlisting}
    \begin{figure}[!ht]
			\centering
			\includegraphics[width=\columnwidth]{/home/pranav/Desktop/Signal processing/Fourier/fou}
			\caption{}
			\label{fig:f}
\end{figure}
\item Show that $x(t)$ is periodic and find its period.\\
    \solution:\begin{align}
    &x(t+\frac{1}{2f_0})=A_0\abs{\sin\brak{2\pi f_0 (t+\frac{1}{2 f_0})}}\\
    &=A_0\abs{\sin\brak{2\pi f_0 t+\pi)}}=A_0\abs{\sin\brak{2\pi f_0 t}}=x(t)
    \end{align}
    $\therefore$ x(t) is periodic with period $\frac{1}{2 f_0}$
\end{enumerate}
\section{Fourier Series}
Consider $A_0 =12$ and $f_0 = 50$ for all numerical calculations.
\begin{enumerate}[label=\thesection.\arabic*,ref=\thesection.\theenumi]
\item If
%\cite{proakis_dsp}
\begin{align}
	x(t) = \sum_{k = -\infty}^{\infty}c_ke^{\j2\pi kf_0 t}
\label{eq:one-Z-complex}
\end{align}
show that 
\begin{align}
	c_k = f_0\int_{-\frac{1}{2f_0}}^{\frac{1}{2f_0}}x(t)e^{-\j2\pi kf_0 t}\, dt
\label{eq:one-Z}
\end{align}
\solution We have for some $n \in \mathbb{Z}$,
\begin{align}
    x(t)e^{-\j2\pi nf_0t} = \sum_{k = -\infty}^{\infty}c_ke^{\j2\pi (k - n)f_0 t}
\end{align}
But we know from the periodicity of $e^{\j2\pi kf_0t}$,
\begin{align}
    \int_{-\frac{1}{2f_0}}^{\frac{1}{2f_0}}e^{\j2\pi kf_0 t}\, dt = 
    \frac{1}{f_0}\delta\brak{k} 
\end{align}
Thus,
\begin{align}
    \int_{-\frac{1}{2f_0}}^{\frac{1}{2f_0}}x(t)e^{-\j2\pi nf_0 t}\, dt = 
    \frac{c_n}{f_0} \\
    \implies c_n = f_0\int_{-\frac{1}{2f_0}}^{\frac{1}{2f_0}}x(t)e^{-\j2\pi nf_0 t}\, dt 
\end{align}
	\item Find $c_k$ for 
	\eqref{eq:10-orig-diff-def}\\
	\solution Using \eqref{eq:one-Z},
\begin{align}
    c_n &= f_0\int_{-\frac{1}{2f_0}}^{\frac{1}{2f_0}}A_0\abs{\sin\brak{2\pi f_0t}}
    e^{-\j2\pi nf_0t}\, dt \\
        &= f_0\int_{-\frac{1}{2f_0}}^{\frac{1}{2f_0}}A_0\abs{\sin\brak{2\pi f_0t}}
    \cos\brak{2\pi nf_0t}\, dt \nonumber \\
        &+ \j f_0\int_{-\frac{1}{2f_0}}^{\frac{1}{2f_0}}A_0
        \abs{\sin\brak{2\pi f_0t}}\sin\brak{2\pi nf_0t}\, dt \\
        &= 2f_0\int_{0}^{\frac{1}{2f_0}}A_0\sin\brak{2\pi f_0t}\cos\brak{2\pi nf_0t}\, dt \\
        &= f_0A_0\int_{0}^{\frac{1}{2f_0}}\brak{\sin\brak{2\pi\brak{n+1}f_0t}}\, dt \nonumber \\ 
        &- f_0A_0\int_{0}^{\frac{1}{2f_0}}\brak{\sin\brak{2\pi\brak{n-1}f_0t}}\, dt \\ 
        &= A_0\frac{1+\brak{-1}^n}{2\pi}\brak{\frac{1}{n+1} - \frac{1}{n-1}} \\
        &= 
        \begin{cases}
            \frac{2A_0}{\pi\brak{1-n^2}} & n\ \text{even} \\
            0 & n\ \text{odd}
        \end{cases}
\end{align}
\item Verify 
	\eqref{eq:10-orig-diff-def}
	using python.\\
	\solution
				Run the following code 
\begin{lstlisting}
wget https://github.com/Pranavb060504/SIgnalProcessing/blob/main/charger/codes/f.py
\end{lstlisting}
Use the following command in the terminal to run the code
\begin{lstlisting}
python3 f.py
\end{lstlisting}
		    \begin{figure}[!ht]
			\centering
			\includegraphics[width=\columnwidth]{/home/pranav/Desktop/Signal processing/Fourier/2_3}
			\caption{}
			\label{fig:f1}
\end{figure}
	\item Show that 
\begin{align}
	x(t) = \sum_{k = 0}^{\infty}\brak{a_k\cos{\j2\pi kf_0 t}+b_k\sin{\j2\pi kf_0 t}}
\label{eq:one-Z-real}
\end{align}
and obtain the formulae for $a_k$ and $b_k$.\\
\solution From \eqref{eq:one-Z-complex},
\begin{align}
    x(t) &= \sum_{k = -\infty}^{\infty}c_ke^{\j2\pi kf_0 t} \\
         &= c_0 + \sum_{k = 1}^{\infty}c_ke^{\j2\pi kf_0t} + c_{-k}e^{-\j2\pi kf_0t} \\
         &= c_0 + \sum_{k = 1}^{\infty}\brak{c_k + c_{-k}}\cos\brak{2\pi kf_0t}  \nonumber \\
         &+ \sum_{k = 0}^{\infty}\brak{c_k - c_{-k}}\sin\brak{2\pi kf_0t}
\end{align}
Hence, for $k \ge 0$,
\begin{align}
    a_k &= 
    \begin{cases}
        c_0 & k = 0 \\
        c_k + c_{-k} & k > 0
    \end{cases} \\
    b_k &= c_k - c_{-k}
    \label{eq:akbk}
\end{align}
\item Find $a_k$ and $b_k$ for 
	\eqref{eq:10-orig-diff-def}\\
	\solution $\because$  $x(t)$ is even,
\begin{align}
    x(-t) &= \sum_{k = -\infty}^{\infty}c_ke^{-\j2\pi kf_0 t} \\
          &= \sum_{k = -\infty}^{\infty}c_{-k}e^{\j2\pi kf_0t} \label{eq:sub} \\
          &= \sum_{k = -\infty}^{\infty}c_ke^{\j2\pi kf_0 t}
\end{align}
where we substitute $k \mapsto -k$ in \eqref{eq:sub}. Hence, we see that 
$c_k = c_{-k}$. So, from \eqref{eq:akbk} and for $k \ge 0$,
\begin{align}
    a_k &= 
    \begin{cases}
        \frac{2A_0}{\pi} & k = 0 \\
        \frac{4A_0}{\pi\brak{1 - k^2}} & k > 0,\ k\ \text{even} \\
        0 & \text{otherwise}
    \end{cases} \\
    b_k &= 0
    \label{eq:akbk-even}
\end{align}
\item Verify 
\eqref{eq:one-Z-real}
using python.\\
\solution
			Run the following code 
\begin{lstlisting}
wget https://github.com/Pranavb060504/SIgnalProcessing/blob/main/charger/codes/f.py
\end{lstlisting}
Use the following command in the terminal to run the code
\begin{lstlisting}
python3 f.py
\end{lstlisting}
	    \begin{figure}[!ht]
			\centering
			\includegraphics[width=\columnwidth]{/home/pranav/Desktop/Signal processing/Fourier/2_6}
			\caption{}
			\label{fig:f1}
\end{figure}
\end{enumerate}
\section{Fourier Transform}
 
\begin{enumerate}[label=\thesection.\arabic*
,ref=\thesection.\theenumi]
\item 
	\begin{align}
		\delta(t)&=0, \quad t\neq0
\\
		\int_{-\infty}^{\infty}\delta(t) \, dt&= 1
	\end{align}
 \item The Fourier Transform of $g(t)$ is
 \begin{align}
 G(f)=\int_{-\infty}^{\infty}g(t)e^{-j2\pi ft}\,dt
 \end{align}
 \item Show that 
 \begin{align}
	 g(t-t_0)&\system{F}G(f)e^{-j2\pi ft_0}\\
 \end{align}
 \solution 
 \begin{align}
     &g(t-t_0)\system{F}\int_{-\infty}^{\infty}g(t-t_0)e^{-j2\pi ft}dt\\
    &\text{Let}\quad t-t_0=k\\
    &\implies g(t-t_0)\system{F}\int_{-\infty}^{\infty}g(k)e^{-j2\pi f(k+t_0)} dk\\
    &=G(f)e^{-j2\pi ft_0}
 \end{align}
 \item Show that 
 \begin{align}
	 G(t)&\system{F}g(-f)
 \end{align}
 \solution Using the definition of the Inverse Fourier Transform,
\begin{align}
    g(t)=\int_{-\infty}^{\infty}G(f)e^{\j2\pi ft}\,df
\end{align}
 $t =-f$ and $f = t$, which implies $df = dt$,
\begin{align}
    g(-f)&=\int_{-\infty}^{\infty}G(t)e^{-\j2\pi ft}\,dt \\
    \implies G(t)&\system{F}g(-f)
\end{align}
 \item $\delta(t)\system{F}?$\\
     \solution \begin{align}
         &=\int_{-\infty}^{\infty} \delta(t) e^{-j2\pi ft}dt\\
         &=\int_{-\infty}^{\infty}\delta(t) e^{-j2\pi f(0)}dt=1
     \end{align}
 \item $e^{-j2\pi f_0t}\system{F}?$\\
     \solution \begin{align}
         &\delta(t-t_0)\system{F}e^{-j2\pi f t_0}\\
         &G(t)\system{F}g(-f)\\
         &\therefore e^{-j2 \pi f_0 t}\system{F}\delta(f-f_0)
     \end{align}
 \item $\cos(2\pi f_0t)\system{F}?$\\
\solution \begin{align}
    &\cos(2\pi f_0t)=\frac{e^{j2\pi f_0t}+e^{-j2\pi f_0t}}{2}\\
    &\cos(2\pi f_0 t)\system{F}\mathcal{F}\left[\frac{e^{j2\pi f_0t}+e^{-j2\pi f_0t}}{2}\right]\\
    &=\frac{\delta(f-f_0)+\delta(f+f_0)}{2}
\end{align}
 \item Find the Fourier Transform of $x(t)$ and plot it.  Verify using python.\\
     \solution
	Run the following code 
\begin{lstlisting}
wget https://github.com/Pranavb060504/SIgnalProcessing/blob/main/charger/codes/3_8.py
\end{lstlisting}
Use the following command in the terminal to run the code
\begin{lstlisting}
python3 3_8.py
\end{lstlisting}
	    \begin{figure}[!ht]
			\centering
			\includegraphics[width=\columnwidth]{/home/pranav/Desktop/Signal processing/Fourier/3_8}
			\caption{}
			\label{fig:3_8}
\end{figure}

 \item Show that 
 \begin{align}
	 \rect{t} \system{F} \sinc{t}
 \end{align}
 Verify using python.\\

 \solution 	Run the following code 
\begin{lstlisting}
wget https://github.com/Pranavb060504/SIgnalProcessing/blob/main/charger/codes/3_9.py
\end{lstlisting}
Use the following command in the terminal to run the code
\begin{lstlisting}
python3 3_9.py
\end{lstlisting}
	    \begin{figure}[!ht]
			\centering
			\includegraphics[width=\columnwidth]{/home/pranav/Desktop/Signal processing/Fourier/3_9}
			\caption{}
			\label{fig:3_9}
\end{figure}

 \begin{align}
     &\rect{t}=\begin{cases}
         1 & \frac{-1}{2}\leq t \leq \frac{1}{2}\\
         0 & \text{else}
     \end{cases}\\
     &\therefore \rect{t}\system{F} \int_{-\infty}^{\infty} \rect{t} e^{-j2\pi ft} dt\\
     &=\int_{\frac{-1}{2}}^{\frac{1}{2}} e^{-j2 \pi ft}dt=\frac{1}{2 \pi jf}\left[ e^{\pi f j}-e^{-\pi f j}\right]\\
     &=\frac{\sin{\pi f}}{\pi f}=\sinc{f}
 \end{align}
 \item 
$	 \sinc{t}\system{F} $?.  Verify using python.\\
\solution	Run the following code 
\begin{lstlisting}
wget https://github.com/Pranavb060504/SIgnalProcessing/blob/main/charger/codes/3_10.py
\end{lstlisting}
Use the following command in the terminal to run the code
\begin{lstlisting}
python3 3_10.py
\end{lstlisting}
	    \begin{figure}[!ht]
			\centering
			\includegraphics[width=\columnwidth]{/home/pranav/Desktop/Signal processing/Fourier/3_10}
			\caption{}
			\label{fig:3_10}
\end{figure}
\begin{align}
    &\because \rect{t}\system{F}\sinc{f}\\
    &\therefore \sinc{t} \system{F} \rect{-f}\\
    &\rect{-f}=\rect{f}
    \end{align}
\end{enumerate}
\section{Filter}
\begin{enumerate}[label=\thesection.\arabic*
,ref=\thesection.\theenumi]
\item Find $H(f)$ which transforms $x(t)$ to DC 5V.\\
\solution $H(f)$ is a low pass filter, assume $f_0$ to be the cutoff frequency,
\\
$H(f)=k \rect{\frac{f}{2f_0}}$,where k is the scaling factor.\\
this is a low pass filter $\because$ all frequencies above $f_0$ are cut-off.\\
now $k=\frac{5\pi}{24}$ can be obtained, $\because$ output voltage$=5 V$.
\item Find $h(t)$.\\
\solution \begin{align}
&\mathcal{F}^{-1}[H(f)]=h(t)\\
& g(at)\system{F}\frac{1}{a}G\left(\frac{f}{a}\right)\\
&\therefore h(t)=2kf_0 \sinc{2f_0 t}
\end{align}
\item Verify your result using  through convolution.
	\solution 	Run the following code 
\begin{lstlisting}
wget https://github.com/Pranavb060504/SIgnalProcessing/blob/main/charger/codes/4_3.py
\end{lstlisting}
Use the following command in the terminal to run the code
\begin{lstlisting}
python3 4_3.py
\end{lstlisting}
	    \begin{figure}[!ht]
			\centering
			\includegraphics[width=\columnwidth]{/home/pranav/Desktop/Signal processing/Fourier/4_3}
			\caption{}
			\label{fig:4_3}
\end{figure}

\end{enumerate}
\section{Filter Design}
\begin{enumerate}[label=\thesection.\arabic*
,ref=\thesection.\theenumi]
\item Design a Butterworth filter for $H(f)$.
	\solution \begin{align}
		|H(f)|=\frac{1}{\sqrt{1+\left(\frac{f}{f_c}\right)^{2n}}}
	\end{align}
		n=order of the filter\\
		wkt $A=10 log_{10}|H(f)|^2$\\
		\begin{align}
			A_1=-10 log_{10} \left[1+\left(\frac{f_1}{f_c}\right)^{2n}\right]\\
A_2=-10 log_{10} \left[1+\left(\frac{f_2}{f_c}\right)^{2n}\right]\\
		\end{align}
		solving for n,
		\begin{align}
			n=\frac{log\left(\frac{10^{-{A_1}/10}-1}{ 10^{-{A_2}/10}-1}\right)}{2 log \left(\frac{f_1}{f_2}\right)}
		\end{align}
		$A_1$=-1 db, $A_2$ =-10 db,$f_1$=50 Hz and $f_2$=100 Hz,we get n=2.5596 $\approx$ 3 \\
		putting n =3 in above equations we get $f_1^{'}$ and $f_2^{'}$ as 62.628 Hz and 69.336 Hz,so we take $f_c=\sqrt{f_1^{'} f_2^{'}}=65.897 Hz$\\
		\item Design a Chebyschev filter for $H(f)$.\\
	\solution \begin{align}
		|H_n(f)|=\frac{1}{\sqrt{1+\epsilon^{2}T_{n}^2\left(\frac{f}{f_0}\right)}}
	\end{align}
where $T_n$=nth order Chebyshev polynomial\\
		$\epsilon$=ripple factor which is related to passband ripple in $\delta$ as $\sqrt{10^{\delta / 10}-1}$\\
\begin{align}
    A_1 = -10\log_{10}\sbrak{1 + \epsilon^2c_n^2\brak{\frac{f}{f_0}}} \\
    \implies c_n\brak{\frac{f}{f_0}} = \frac{\sqrt{10^{-\frac{A_1}{10}} - 1}}{\epsilon} \\
    \implies n = \frac{\cosh^{-1}\brak{\frac{\sqrt{10^{-\frac{A_1}{10}} - 1}}{\epsilon}}}
    {\cosh^{-1}\brak{\frac{f}{f_0}}}
\end{align}
considering $f_0=65 Hz$,$f$=120 Hz ,$A_1=10  dB$,$\delta=0.2 dB$,$\epsilon= 0.217$. on solving we get n=2.71302,$\implies$ n=3
\item Design a circuit for your Butterworth filter.\\
	\solution for this 3rd order filter the $C_1=1,C_3=1$ and $L_2=2$, de-normalizing the values $C_i^{'}=\frac{C_i}{\omega_c}$and similarly  $L_i^{'}=\frac{L_i}{\omega_c}$.
		\begin{figure}[!htb]
    \begin{center}
    \begin{circuitikz} 
    \draw
	(0,0) -- (4,0)--(8,0) 
	to[C, l^=3.18mF] (8,4)
	to[L, l=6.36H] (4,4)
	to[C,l^=3.18F] (4,0)
	    (4,4)--(0,4);
    \end{circuitikz}
    \end{center}
\caption{}
\label{fig:3BF}
\end{figure} 
\item Design a circuit for your Chebyschev filter.\\
	\solution the normalised values obtained are$\quad C_1=1.2276 F,L_2=1.1525 H,C_3=1.2276 F$. De-normalizing the values we get:
		\begin{figure}[!htb]
    \begin{center}
    \begin{circuitikz} 
    \draw
	(0,0) -- (4,0)--(8,0) 
	to[C, l^=1.99mF] (8,4)
	to[L, l=1.876mH] (4,4)
	to[C,l^=1.99mF] (4,0)
	    (4,4)--(0,4);
    \end{circuitikz}
    \end{center}
\caption{}
\label{fig:3CF}
\end{figure}
\end{enumerate}
\end{document}
